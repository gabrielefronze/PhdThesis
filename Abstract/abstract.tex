% ************************** Thesis Abstract *****************************
% Use `abstract' as an option in the document class to print only the titlepage and the abstract.
% \begin{abstract}
% Abstract: ALICE is devoted to the study of a deconfined state of nuclear matter called Quark Gluon Plasma (QGP), in which quarks and gluons behave as free particles. 
% The bottomonium (bound states of beauty-anti beauty quark) production is affected by the presence of the QGP, since bottomonium states are produced sooner than the QGP and witness the whole evolution of the plasma.
% In this analysis the data coming from Pb-Pb collisions have been analyzed in order to detect possible modifications of the production rates in the dimuon decay channel, with respect to the rates observed in proton-proton collisions. 
% Furthermore, the performances of the detectors involved in the muon identification during the LHC RUN1 and RUN2 has been tested using a new analysis framework implemented as part of this thesis. 
% Finally, in order to optimize the results of future analyses, a new muon identification algorithm has been developed and tested. This algorithm will become necessary in the LHC RUN3 running conditions, when the much higher luminosity will require a quasi-online reconstruction of data.


% Résumé: ALICE est dédié à l’étude d’un état de la matière nucléaire dans lequel les quarks et les gluons ne sont plus confinés dans les hadrons, qui est appelé́ Quark Gluon Plasma (QGP).
% La production de bottomonia (états liés beauté́ anti- beauté́ ) est sensible au QGP parce-que les états du bottomonium sont formés avant la formation du QGP et traversent le plasma pendant son évolution. 
% L'objectif principal de cette thèse est la mesure des modification des mésons Upsilon dans le canal de désintégration en deux muons en collisions Pb-Pb à $\sqrt{s_{NN}} = 5\ TeV$.
% En outre, un nouveau framework pour l'analyse des performances des détecteurs utilisés pour l'identification des muons a été realisé et utilisé pour l'analyse des données du RUN1 et RUN2 du LHC.
% Enfin, et avec l’objectif d’optimiser des résultats de l’analyse, un nouvel algorithme d’identification de muons a été développé.
% Cet algorithme deviendra nécessaire pour faire face aux nouvelles conditions de prise de données du RUN3, pendant lequel une reconstitution quasi-en ligne du détecteur est prévue.
% \end{abstract}

\cleardoublepage

\selectlanguage{english}
\renewcommand{\abstractname}{Abstract}
\begin{abstract}
ALICE is devoted to the study of a deconfined state of nuclear matter called Quark Gluon Plasma (QGP).
The bottomonium (bound states of beauty-anti beauty quark) production is affected by the presence of the QGP, since bottomonium states are produced sooner than the QGP and witness the whole evolution of the plasma.
In this analysis the data coming from Pb-Pb collisions at $\sqrt{s_{NN}} = 5\ \rm TeV$ have been analyzed in order to detect possible modifications of the production rates, with respect to the rates observed in proton-proton collisions. 
Furthermore, the performances of the detectors involved in the muon identification during the LHC RUN1 and RUN2 has been tested using a new analysis framework implemented as part of this thesis. 
Finally, in order to optimize the results of future analyses, a new muon identification algorithm has been developed and tested. This algorithm will become necessary in the LHC RUN3 running conditions, when the much higher luminosity will require a quasi-online reconstruction of data.
\end{abstract}

\selectlanguage{french}
\renewcommand{\abstractname}{Résumé}
\begin{abstract}
ALICE est dédié à l’étude d’un état de la matière nucléaire dans lequel les quarks et les gluons ne sont plus confinés dans les hadrons, qui est appelé Quark Gluon Plasma (QGP).
La production de bottomonia (états liés beauté anti- beauté ) est sensible au QGP parce-que les états du bottomonium sont formés avant la formation du QGP et traversent le plasma pendant son évolution. 
L'objectif principal de cette thèse est la mesure des modification des mésons Upsilon en collisions Pb-Pb à $\sqrt{s_{NN}} = 5\ \rm TeV$.
En outre, un nouveau framework pour l'analyse des performances des détecteurs utilisés pour l'identification des muons a été realisé et utilisé pour l'analyse des données du RUN1 et RUN2 du LHC.
Enfin, et avec l’objectif d’optimiser des résultats de l’analyse, un nouvel algorithme d’identification de muons a été développé.
Cet algorithme deviendra nécessaire pour faire face aux nouvelles conditions de prise de données du RUN3, pendant lequel une reconstitution quasi-en ligne du détecteur est prévue.
\end{abstract}

\selectlanguage{english}

\clearpage

\renewcommand{\abstractname}{Plan of the thesis}
\begin{abstract}
The thesis reported in the following is organized as follows:
\begin{enumerate}
    \item In Chapter 1 an introduction to the Quark Gluon Plasma and to the expected suppression of quarkonia resonances in the deconfined medium will be given, along with a brief overview of previous results obtained by RHIC and LHC experiments. In addition, the ALICE experimental setup will be introduced, with particular regard to the muon spectrometer and the other detectors involved in the analysis.
    \item In Chapter 2 a new framework for the study of ALICE muon trigger performances, developed as one of the subjects of this thesis, will be presented. The measurements obtained through the application of newly introduced algorithms will be presented and the status of the Resistive Plate Chambers composing the muon trigger system will be discussed.
    \item In Chapter 3 the measurement of $\Upsilon$ suppression in Pb-Pb collisions at $\sqrt{s_{\mathrm{NN}}}=5.02$ TeV will be presented, with a detailed description of the analysis procedures and the comparison with theoretical models.
    \item In Chapter 4 an introduction to the Online-Offline ($O^2$) ALICE upgrade will be given. Subsequently, a new online processing algorithm for the muon trigger data will be presented and its performances discussed by by means of the results of several tests.
    \item In Chapter 5 some conclusions on the work carried out in the thesis will be presented.
\end{enumerate}
\end{abstract}
\clearpage