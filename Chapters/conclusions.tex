\chapter{Conclusions}
The aim of this thesis is the study of quarkonium production in heavy-ion collisions, as performed with the muon spectrometer of the ALICE experiment at  the Large Hadron Collider. 
Three different aspects have been tackled within the thesis.
First, a study has been performed of the long-term evolution during the LHC RUN1 and RUN2 of a few relevant parameters for aging effects in the muon trigger detectors (Resistive Plate Chambers): the integrated charge, the dark current and the dark rate. 
In order to do this, a new framework was developed, able to combine several sources of data. It has been developed as an extension of the AliROOT \code{Shuttle}, which is in charge of transferring parameters related to the detector and the accelerator to the Offline Condition Database (OCDB). 
The new tool, called \code{MTRShuttle}, allows one to combine the OCDB information with the data from the Detector Control System (DCS). 
The main result of the study is that an increasing trend of dark current is observed for a non-negligible fraction of detectors.
Although this effect is not accompanied by a decrease of the efficiency, the status of these chambers shall be monitored and replacement may be considered.
Further studies to understand the origin of the dark current are also recommended.
In view of the increased luminosity of the LHC RUN3, one RPC has been equipped with a prototype for an amplified front-end electronics, allowing for low-gain operation of the RPC. 
The study of its parameters has shown a more stable dark rate and dark current, and a much smaller integrated charge, which is a promising result. 
The functionalities of the developed framework may be extended to other ALICE systems in the future.
Second, the study of bottomonium production at forward rapidity in Pb-Pb collisions at $\sqrt{s_{\mathrm{NN}}}=5.02\ TeV$, in the dimuon decay channel, has been performed.
Data collected during the LHC RUN2 were analysed, and the nuclear modification factor of the $\Upsilon(1S)$ and $(2S)$ were measured, showing a clear suppression with respect to the properly scaled reference obtained in pp collisions. 
The suppression is larger for the $\Upsilon(2S)$, as expected in the hypothesis of a binding-energy dependent sequential suppression in a Quark Gluon Plasma. 
For the $\Upsilon(1S)$, the measurement was also performed as a function of centrality, rapidity and transverse momentum. 
While the suppression increases with the collision centrality, no significant trend as a function rapidity and transverse momentum is observed within the present uncertainties. 
The measured nuclear modification factors are also compatible with those measured during the LHC Run1 at $\sqrt{s_{\mathrm{NN}}}=2.76\ TeV$.
The results were compared to transport and dynamical model calculations, which qualitatively reproduce the centrality, rapidity and transverse momentum dependence.
The paper "$\Upsilon$ suppression at forward rapidity in Pb-Pb collisions at $\sqrt{s_{\mathrm{NN}}} = 5.02 TeV$"\cite{MINE} was drawn up and is currently undergoing the publication \textit{iter} for the Physics Letters B journal.
Third, a new muon identification software has been developed in light of the RUN3 upgrade of ALICE, and the associated software upgrade program called $O^2$.
With this upgrade the data rate will go from $8\ kHz$ to $50\ kHz$ of minimum bias events in Pb-Pb collisions.
Thanks to online processing of all the $50\ kHz$ of events, the limit storage bandwidth will be saturated with interesting events.
The ALICE muon trigger (MTR) system will undergo a big upgrade process, involving the read out electronics and the whole software apparatus, allowing it to efficiently serve as a muon identifier (MID).
In the context of this work a big effort has been put to develop a fast, precise and resilient software capable of coping with the RUN 3. 
As a result the ALICE MTR/MID will be able to retain most of its features and performances along with a highly improved rate capability. 
The development is still work in progress, but the boost given by this work is an important fraction of the planned tasks.
A library, which allows for the transport of \code{C++} types and structures (and contiguous containers of such objects) using the ZeroMQ transport framework at the basis of $O^2$, has been implemented.
These serialization routines, based on Boost have been widely tested and are currently used in many detector- and subsystem-specific code, and was requested to be included in FairROOT core by its developers team. 
The development of the $O^2$ software for the MID involved several work items.
An algorithm capable of de-serializing the CRU data format in a $C++$ structure has been implemented.
Such algorithm will be probably implemented directly on the CRU FPGA for timing constraints.
In addition, the development of a full stack of devices, capable of performing an online rejection of MID noisy channels and detection of dead channels, has been accomplished.
This chain of devices has been implemented on top of the Data Processing Layer (DPL) framework.
The interaction with DPL developers allowed for an improvement of the whole framework by looking to the real use case of the MID detector.
Finally, preliminary tests of the performance of the new muon reconstruction algorithm for MID have been carried out, with promising results.
