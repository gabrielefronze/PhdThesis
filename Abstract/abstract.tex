% ************************** Thesis Abstract *****************************
% Use `abstract' as an option in the document class to print only the titlepage and the abstract.
% \begin{abstract}
% Abstract: ALICE is devoted to the study of a deconfined state of nuclear matter called Quark Gluon Plasma (QGP), in which quarks and gluons behave as free particles. 
% The bottomonium (bound states of beauty-anti beauty quark) production is affected by the presence of the QGP, since bottomonium states are produced sooner than the QGP and witness the whole evolution of the plasma.
% In this analysis the data coming from Pb-Pb collisions have been analyzed in order to detect possible modifications of the production rates in the dimuon decay channel, with respect to the rates observed in proton-proton collisions. 
% Furthermore, the performances of the detectors involved in the muon identification during the LHC RUN1 and RUN2 has been tested using a new analysis framework implemented as part of this thesis. 
% Finally, in order to optimize the results of future analyses, a new muon identification algorithm has been developed and tested. This algorithm will become necessary in the LHC RUN3 running conditions, when the much higher luminosity will require a quasi-online reconstruction of data.


% Résumé: ALICE est dédié à l’étude d’un état de la matière nucléaire dans lequel les quarks et les gluons ne sont plus confinés dans les hadrons, qui est appelé́ Quark Gluon Plasma (QGP).
% La production de bottomonia (états liés beauté́ anti- beauté́ ) est sensible au QGP parce-que les états du bottomonium sont formés avant la formation du QGP et traversent le plasma pendant son évolution. 
% L'objectif principal de cette thèse est la mesure des modification des mésons Upsilon dans le canal de désintégration en deux muons en collisions Pb-Pb à $\sqrt{s_{NN}} = 5\ TeV$.
% En outre, un nouveau framework pour l'analyse des performances des détecteurs utilisés pour l'identification des muons a été realisé et utilisé pour l'analyse des données du RUN1 et RUN2 du LHC.
% Enfin, et avec l’objectif d’optimiser des résultats de l’analyse, un nouvel algorithme d’identification de muons a été développé.
% Cet algorithme deviendra nécessaire pour faire face aux nouvelles conditions de prise de données du RUN3, pendant lequel une reconstitution quasi-en ligne du détecteur est prévue.
% \end{abstract}

\cleardoublepage

\selectlanguage{english}
\renewcommand{\abstractname}{Abstract}
\begin{abstract}
ALICE is devoted to the study of a deconfined state of nuclear matter called Quark Gluon Plasma (QGP).
The bottomonium (bound states of beauty-anti beauty quark) production is affected by the presence of the QGP, since bottomonium states are produced sooner than the QGP and witness the whole evolution of the plasma.
In this analysis the data coming from Pb-Pb collisions at $\sqrt{s_{NN}} = 5\ \rm TeV$ have been analyzed in order to detect possible modifications of the production rates, with respect to the rates observed in proton-proton collisions. 
Furthermore, the performances of the detectors involved in the muon identification during the LHC RUN1 and RUN2 has been tested using a new analysis framework implemented as part of this thesis. 
Finally, in order to optimize the results of future analyses, a new muon identification algorithm has been developed and tested. This algorithm will become necessary in the LHC RUN3 running conditions, when the much higher luminosity will require a quasi-online reconstruction of data.
\end{abstract}

\selectlanguage{french}
\renewcommand{\abstractname}{Résumé}
\begin{abstract}
ALICE est dédié à l’étude d’un état de la matière nucléaire dans lequel les quarks et les gluons ne sont plus confinés dans les hadrons, qui est appelé Quark Gluon Plasma (QGP).
La production de bottomonia (états liés beauté anti- beauté ) est sensible au QGP parce-que les états du bottomonium sont formés avant la formation du QGP et traversent le plasma pendant son évolution. 
L'objectif principal de cette thèse est la mesure des modification des mésons Upsilon en collisions Pb-Pb à $\sqrt{s_{NN}} = 5\ \rm TeV$.
En outre, un nouveau framework pour l'analyse des performances des détecteurs utilisés pour l'identification des muons a été realisé et utilisé pour l'analyse des données du RUN1 et RUN2 du LHC.
Enfin, et avec l’objectif d’optimiser des résultats de l’analyse, un nouvel algorithme d’identification de muons a été développé.
Cet algorithme deviendra nécessaire pour faire face aux nouvelles conditions de prise de données du RUN3, pendant lequel une reconstitution quasi-en ligne du détecteur est prévue.
\end{abstract}

\selectlanguage{english}

\clearpage

\renewcommand{\abstractname}{Plan of the thesis}
\begin{abstract}
The thesis reported in the following is organized as follows:
\begin{enumerate}
    \item In Chapter 1 an introduction to the Quark Gluon Plasma and to the expected suppression of quarkonia resonances in the deconfined medium will be given, along with a brief overview of previous results obtained by RHIC and LHC experiments. In addition, the ALICE experimental setup will be introduced, with particular regard to the muon spectrometer and the other detectors involved in the analysis.
    \item In Chapter 2 a new framework for the study of ALICE muon trigger performances, developed as one of the subjects of this thesis, will be presented. The measurements obtained through the application of newly introduced algorithms will be presented and the status of the Resistive Plate Chambers composing the muon trigger system will be discussed.
    \item In Chapter 3 the measurement of $\Upsilon$ suppression in Pb-Pb collisions at $\sqrt{s_{\mathrm{NN}}}=5.02$ TeV will be presented, with a detailed description of the analysis procedures and the comparison with theoretical models.
    \item In Chapter 4 an introduction to the Online-Offline ($O^2$) ALICE upgrade will be given. Subsequently, a new online processing algorithm for the muon trigger data will be presented and its performances discussed by by means of the results of several tests.
    \item In Chapter 5 some conclusions on the work carried out in the thesis will be presented.
\end{enumerate}
\end{abstract}
\clearpage

\selectlanguage{french}

\clearpage

\renewcommand{\abstractname}{Summaire}
\begin{abstract}
ALICE est dediée à l’étude d’un état de la matière nucléaire dans lequel les quarks et les gluons ne sont plus confinés dans les hadrons, qui est appelé Quark Gluon Plasma (QGP).
La production de bottomonia (états liés beauté anti-beauté) est sensible au QGP parce-que les états du bottomonium sont formé avant la formation du QGP et sont ainsi capables de traverser le plasma pendant son évolution.
Pendant cette thèse l’analyse des données de collisions Pb-Pb et pp sera poursuivie afin d’étudier les modifications de la production du méson Upsilon dans le canal de désintégration en deux muons. 
En outre, et avec l’objectif d’optimisation des résultats de l’analyse, un nouvel algorithme d’identification de muons sera développé, pour se préparer aux spécifications pour le troisième run de LHC, dans lequel les spécifiques de fonctionnement de l’accélérateur exigeront une reconstruction quasi-online.


L’objectif principal de la première année du doctorat etait l’analyse des mésons Upsilon en collisions Pb-Pb collecté fin 2015 etant membre du groupe d’analyse du méson Upsilon d’ALICE. 
Dans ce cadre, l’estimation de l’incertitude systématique due à la réponse du système de déclenchement du spectromètre à muon etait l'argument principal de ce travail de thése.
Pour s'assurer des performances de ce système, un ètude des performances du muon trigger system pendant la dernière période de prise de données (2015) en collisions pp et Pb-Pb etait achevé. 
Les résultats de ce travail ont été presenté à la conférence RPC2016 à Ghent (Belgique).


Pendent la deuxieme année de doctorat le travail d'analyse continué avec la finalisation des incertitudes systématiques.
En ajoute l’évaluation de la section efficace de production du méson upsilon en utilisant les données de LHCb etait achévée.
Pour étudier les modifications de la section efficace de production des états du quarkonium on doit calculer l’$R_{\mathrm{AA}}$ ou facteur de modification nucléaire.
Dans ce facteur apparait le nombre de états reconstruit dans les évènements, divisé par la section efficace de production du même état dans collisions proton-proton.
La statistique disponible en collision pp à $5$ TeV par ALICE n’était pas suffisante pour calculer la section efficace.
Pour cette raison une interpolation des valeurs de section efficace mesurés par LHCb a différente énergies ($2.76$, $7$ et $8$ TeV) etait necessaire pour obtenir la valeur pour le calcul du facteur de modification nucléaire a $5.02$ TeV. 
En suite au travail de la premiere année, à partir des derniers mois du 2016, une suite de logiciels pour l’analyse automatique des paramètres de fonctionnement des chambre RPC du système de déclenchement à muons etait developpée.
Ce systeme etait developpé sur la base de l'experience aquise au debut du doctorat.
Ce système comprend un protocole de téléchargement et analyse des données très facile à utiliser.
Cet aspect avec la définition d’une manière de decrire et créer automatiquement les histogrammes et les graphs nécessaires pour l’analyse iront changer et rendre plus rapide l’étude de la condition opérative du système de déclenchement à muons.


La derniere partie du travail the these porte sur le développement du système de trajectographie en ligne pour le système de déclenchement du spectromètre à muon.
Ce developpement est necessaire pour rendre ALICE capable de travailler avec les nouvelles conditions présentes dans les prochaines phase de récolte des donnés.
Sera, donc, nécessaire l’implémentation de plusieurs nouveaux devices pour adresser plusieurs fins. 
En premier lieu un device pour le streaming des données actuelles traduits dans le format des nouvelles données doit être développe pour permettre le test des autres devices dans conditions de travail le plus proches possible à celles qui seront montré dans le RUN3 d’ALICE. 
Le deuxième device qui sera développe passera à la tâche de l’exécution de l’algorithme de trajectographie dans le système de déclanchement de muons. 
Ce device doit être écrit en utilisant paradigmes de computation a très hautes performances pour permettre la reconstruction et l’identification des muons presque online pour permettre la sélection des événements. 
En particulier le travail ici presenté concerne un set de logiciels indépendants qui peuvent masquer en ligne les canaux bruyants du système de déclanchement etait developpé et, plus en général, calculer en ligne la fraction allumée du détecteur.
\end{abstract}

\selectlanguage{english}